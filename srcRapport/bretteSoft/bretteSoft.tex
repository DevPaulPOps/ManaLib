\chapter*{BretteSoft}
\addcontentsline{toc}{chapter}{BretteSoft}

Pour la partie BretteSoft, nous avons implémenté une seule fonctionnalité : \textbf{Sitting Bull}.
Cette fonctionnalité permet de notifier un utilisateur par courriel lorsque ce dernier souhaite emprunter
un document déjà réservé.

\section{Implementation}

\bigskip

Pour implémenter cette fonctionnalité, nous avons intégré dans un fichier existant nommé \textbf{Utils} les
fonctions nécessaires pour envoyer un courriel à une personne choisie, en l'occurrence le grand Wakan Tanka,
à l'adresse suivante : \textbf{jean-francois.brette@u-paris.fr}. Cela signifie que lorsqu'une personne souhaite réserver un document,
il lui sera demandé si elle souhaite être notifiée. Si elle répond positivement, un courriel lui sera envoyé dès que
le document sera disponible.

\begin{itemize}
    \item \textbf{Javax.mail} : Bibliotheque utilisé pour envoyer des mails.
\end{itemize}

Concernant la sécurité, les paramètres du courriel, notamment l'adresse d'envoi, le port ainsi que le mot de passe,
ont été stockés dans un fichier // d'environnement (a changer ???) //.

\section{Processus}

\bigskip

Pour faire fonctionner cette bibliothèque, nous devons d'abord mettre en place certains éléments, notamment le mot
de passe pour l'accès aux applications. En effet, pour pouvoir se connecter et ensuite envoyer un courriel à une personne,
il est nécessaire de générer un mot de passe pour l'authentification. Auparavant, il suffisait de cocher une case dans la
section "Less secure App" de Google, mais cette option a été supprimée depuis le 30 septembre 2024.

\bigskip

Sources :

\begin{itemize}
    \item \textbf{Envoyer mail SMTP} : \href{https://kb.synology.com/fr-fr/SRM/tutorial/How_to_use_Gmail_SMTP_server_to_send_emails_for_SRM}{Gmail SMTP server to send emails for SRM}
    \item \textbf{Less secure App} : \href{https://support.google.com/accounts/answer/6010255?sjid=6174101903992956370-EU}{Google Accounts Answer}
\end{itemize}

\bigskip



\bigskip

