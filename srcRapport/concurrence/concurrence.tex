\chapter*{Concurrence}
\addcontentsline{toc}{chapter}{Concurrence}

    Nous avons utilisé des blocs “synchronized” ainsi que des fonctions pour protéger les sections critiques de notre application comme pour le
    paradigme du problème des philosophes afin d’assurer une exécution thread-safe. La concurrence se trouve entre les
    différentes opérations (réservation, retour, emprunt) car elles ne peuvent être exécutées simultanément sur un même document.

    \bigskip

    C’est en soit intuitif mais il est important de l’implémenter correctement pour éviter les conditions de course.
    Les sections synchronisées sont donc exécutées que lorsqu’elles sont disponibles ceci garantie le fait que l’application
    soit thread-safety.

    \bigskip

