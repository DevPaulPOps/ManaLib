\chapter*{Maintenance évolutive}
\addcontentsline{toc}{chapter}{Maintenance évolutive}

\section{Ajout de nouveaux Documents}

\bigskip

L’ajout de nouveau Document nécessite d’implémenter l’interface correspondant aux Document ainsi que son
Modèle respectif si l'on veut le stocker dans la base de données.
La logique des opérations de Réservation, emprunt, retour ne changeront pas. C’est du \textbf{\textcolor{orange}{Pattern Bridge}} que nous nous sommes
inspirés pour implémenter les Documents.

\bigskip

\section{Passage en application web}

\bigskip

Pour passer sur une application Web nous devons revoir notre serveur de la médiathèque. Les opérations ne changeront pas mais
toute notre partie serveur devra être revu. Nous devons donc optimiser la maintenabilité de l’application afin de dissocier la médiathèque Server et son service du protocole de communication existant.

\bigskip

\section{Ajout nouveaux services}

\bigskip

La numérotation des ports ainsi que les urls sont géré par une class config qui lit un fichier config.json.
Nous avons utilisé ceci car c’est une bonne pratique en terme de sécurité de ne pas écrire en clair dans le code les urls port.

Donc, pour ajouter un nouveau service, il suffit d'attribuer les informations confidentielles à ce service dans la partie configuration, puis d'implémenter la classe du service. Il est important de ne pas oublierd'implémenter l'interface correspondante à la médiathèque, qui est propre à la relation entre le service et le serveur.

