\chapter{Maintenance évolutive}


\section{Ajout de nouveaux Documents}

L’ajout de nouveaux documents nécessite d’implémenter l’interface correspondant aux documents ainsi que son modèle respectif si l'on veut le stocker dans la base de données. La logique des opérations de réservation, emprunt, et retour ne changera pas. C’est du \textbf{\textcolor{orange}{Pattern Bridge}} que nous nous sommes inspirés pour implémenter les documents.


\section{Passage en application web}

Pour passer sur une application Web, nous devons revoir notre serveur de la médiathèque. Les opérations ne changeront pas mais toute notre partie serveur devra être revue. Nous devons donc optimiser la maintenabilité de l’application afin de dissocier la médiathèque Server et son service du protocole de communication existant.


\section{Ajout de nouveaux services}

La numérotation des ports ainsi que les URL sont gérées par une classe \texttt{Config} qui lit un fichier \texttt{config.json}. Nous avons utilisé ceci car c’est une bonne pratique en termes de sécurité de ne pas écrire en clair dans le code les URL et ports.

Pour ajouter un nouveau service, il suffit d'attribuer les informations confidentielles à ce service dans la partie configuration, puis d'implémenter la classe du service. Il est important de ne pas oublier d'implémenter l'interface correspondante à la médiathèque, qui est propre à la relation entre le service et le serveur.
